\documentclass[12pt]{article}

\usepackage{vmargin}
\setpapersize{A4}
\setmarginsrb{3cm}{2cm}{1.5cm}{2cm}{0pt}{10mm}{0pt}{10mm}

\usepackage{moreverb}
\usepackage[utf8]{inputenc}
\usepackage[T1,T2A]{fontenc}
\usepackage[russian]{babel}
\usepackage{vmargin}
\usepackage{amsmath}
\usepackage{amssymb}
\usepackage{graphicx}
\usepackage{amsthm} 
\graphicspath{{img/}}
\usepackage{verbatim}
\usepackage{listings}
\usepackage{hyperref}
\setcounter{tocdepth}{2}
\title{Стохастический градиентный спуск и бета-регрессия}
\author{Олейник Михаил, группа 24.М22-мм}
\date{\today}
\begin{document}
	\maketitle
	
	Весь код с комментариями и результатами можно найти в \href{https://github.com/UnicornRed/GradientDescentBeta}{репозитории}.
	
	\section{Задача линейной регрессии}
	
	Решение задачи линейной регрессии заключается в нахождении коэффициентов $\overline\beta$ линейной комбинации для случайной величины $\xi \sim N(\mu _\xi, \sigma)$ и случайного вектора $\overline{\eta} \sim \mathbf{N}(\overline \mu _\eta, \mathbf{\Sigma})$:
	\[
	\xi = \overline\eta ^T \overline\beta + \varepsilon,
	\]
	
\end{document}